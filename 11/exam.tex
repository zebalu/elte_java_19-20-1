\documentclass[12pt,a4paper]{article}
\usepackage[utf8]{inputenc}
\usepackage[magyar]{babel}
\usepackage[T1]{fontenc}
\usepackage{amsmath}
\usepackage{amsfonts}
\usepackage{amssymb}
\usepackage{hyperref}

\begin{document}


\section{Általános tudnivalók}

Ebben az ismertetésben az osztályok, valamint a minimálisan szükséges
metódusok leírásai fognak szerepelni.  A feladatmegoldás során fontos
betartani az elnevezésekre és típusokra vonatkozó megszorításokat, illetve
a szövegek formázási szabályait. \\

Nem publikus láthatóságú segédfüggvények létrehozhatóak, a feladatban nem
megkötött adattagok és elnevezéseik is a feladat megoldójára vannak bízva.
Törekedjünk arra, hogy az osztályok belső reprezentációját a *lehető
legjobban védjük*, tehát csak akkor engedjünk, és csak olyan hozzáférést,
amelyre a feladat felszólít, vagy amit azt osztályt használó kódrészlet
megkíván! \\ 

Használható segédanyagok:
\href{https://bead.inf.elte.hu/files/java/api/index.html}{Java dokumentáció} \\
legfeljebb egy üres lap és toll.  Ha bármilyen kérdés, észrevétel felmerül,
azt a felügyelőknek kell jelezni, *NEM* a diáktársaknak!


\section{Feladat leírás}

Készítsük el az Egyesült Prognyelvek községnek a parlamenti választórendszerét. 
Az elkészített osztályokat helyezzük a \textit{valasztas} csomagba, és hozzunk 
létre minden osztálynak egy külön fordítási egységet. \\
 
A szavazó polgárokról nyilván tartjuk a nevüket (karakterlánc), 
azonosítójuk (karakterlánc) életkorukat (egész szám, mely nem lehet negatív), 
országkódjukat (egész szám). A polgárok adattagjai legyenek publikusan 
lekérdezhetőek és beállíthatóak. Hozzunk létre egy publikus konstruktort az 
osztálynak, mely beállítja a kezdeti adattagokat \\ 
A szavazó polgárok három különböző pártok szavazhatnak 
(Felsorolási típus: JavaHősök, CppFanok, HaskellMágusok). \\

\begin{enumerate}


\item Kettes szerint: A pártokra egy szavazórendszeren keresztül szavazhatnak, 
ahol nyilván tartjuk a leadott szavazatokat. A leadott szavazatokat egy 
\textit{ArrayList} adatszerkezetben tartsuk nyilván, ennek megfelelően nevezzük 
el \textit{ArraySzavazorendszer}-nek az osztályunkat.

\textit{ArraySzavazorendszer} publikus metódusai a következők:
\begin{itemize}
\item \textit{szavaz}: Paraméterként a szavazópolgárt, valamint egy pártot vár. 
Bejegyez egy új szavazatot a megadott pátra.
\item \textit{osszSzavazat}: Megadja, hogy összesen eddig hány polgár szavazott.
\item \textit{lezar}: Lezárja a szavazást, mely után már nem lehet szavazni.
\item \textit{nyertes}: Visszaadja, melyik párt a nyertes. Ha a szavazás még 
nincs lezárva, dobjunk kivételt.
\item \textit{hanySzavazat}: Egy pártot vár paraméterként, és megadja, hogy egy 
adott pártra összesen eddig hányan szavaztak. Ha a szavazás még nincs lezárva, dobjunk kivételt.
\end{itemize}


\item Hármas szint: Hozzunk létre egy \textit{Azonosithato} interfészt, melynek 
legyen egy \textit{getKulcs} művelete, mely egy karakterlánccal tér vissza. 
A személyiségi jogok megőrzése érdekében módosítsuk a \textit{szavaz} függvényt 
úgy, hogy egy \textit{Azonosithato} típust várjon, így ne férjen hozzá a polgár 
minden adatához. A polgár ennek megfelelően valósítsa meg a \textit{getKulcs} 
műveletet úgy, hogy adja vissza saját azonosítójának és országkódjának a 
konkatenációját. Figyeljünk oda, hogy ugyanazon kulccsal ne lehessen leadni 
szavazatot. Ilyen kísérlet esetén dobjon a \textit{szavaz} függvény egy saját 
\textit{AlreadyVoted} kivételt, melytől le lehet kérdezni a kulcsot, amivel 
megkíséreltünk újra szavazni.

\item Négyes szint: A hatékonyság növelése hozzunk létre egy interfészt a 
\textit{Szavazorendszer}-nek, és készítsünk egy másik implementációt, melyben 
a szavazatokat egy \textit{HashMap}-ben (kulcs-érték párokat tartalmazó adatszerkezet) 
tároljuk. Ezt az implementációt nevezzük \textit{HashSzavazorendszer}-nek, mely 
implementálja a Szavazorendszer interfészt. Az eddig implementált 
\textit{ ArraySzavazorendszer}-t szintén valósítsa meg ezt az újonnan létrehozott interfészt.

\item Ötös szint: Az Azonosithato interfész legyen generikus, melynek típusparamétere 
a \textit{getKulcs} visszatérési értékének típusa legyen. Ennek megfelelően a
\textit{Szavazórendszer} is legyen egy generikus interfész, annak implementációi 
pedig generikus osztályok, melyeknek az adatszerkezetükben tárolt kulcsainak típusa 
a típusparaméter legyen. Csináljuk egy \textit{PolgarKulcs} osztályt, mely tartalmaz 
egy azonosítót és országkódot, és ezzel a típusparaméterrel hozzuk létre a polgárokat 
és a szavazórendszereket. Érjük el, hogy tudjuk kulcsként használni ezt a típust az 
\textit{ArrayList} és \textit{HashMap}-en belül (ehhez írjuk felül az \textit{equals}
és \textit{hashCode} függvényeit az alábbi szabály szerint: ha két objektum egyenlő
egymással, akkor a \textit{hashCode} műveletüknek ugyanazzal a számmal kell visszatérnie).

\end{enumerate}

A \textit{main} függvény hozzon létre három szavazópolgárt, és egy választási rendszert.
Az első és a második polgár szavazzon a JavaHősök pátra, majd az első polgár próbáljon
meg újra szavazni a CppFanokra. A harmadik személy szavazzon a HaskellMágusokra. 
Zárjuk le a szavazást, írjuk ki, hogy összesen hányan szavaztak, és hogy ki a nyertes párt.

\end{document}
